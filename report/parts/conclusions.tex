\chapter{Conclusiones y trabajos futuros}\label{chap:conclusions}
En este capítulo se presentan las conclusiones tras la realización de este proyecto (Sección~\ref{sec:conclussions}). En esta sección se estudiarán los objetivos indicados en el capítulo~\ref{chap:introduction}. Además, se indican una serie de trabajos futuros (Sección~\ref{sec:future-work}) para la mejora de la propuesta.

\section{Conclusiones}\label{sec:conclussions}
En esta sección se hace un estudio de los objetivos propuestos en la sección~\ref{sec:objectives} con el fin de evaluar su cumplimineto.

El primer objetivo indica que se debe crear una nueva arquitectura para el simulador CREATOR que incluya el nuevo conjunto de instrucciones implementado. Este objetivo se ha traducido directamente en un requisito de software (Capítulo \ref{chap:analysis}) que ha sido verificado durante la validación y verificación del proyecto (Capítulo~\ref{chap:validation})). Es por esto que se puede afirmar que este objetivo se ha cumplido.

El segundo objetivo indica que se debe crear un motor de ejecución para dar soporte a operaciones con registros vectorailes. De nuevo, y al igual que en el caso anterior, este requisito se ha traducido a un requisito que ha sido validado en el correspondiente capítulo. Adicionalmente, en el presente documento se han expuesto los componentes y arquitectura (Capítulo~\ref{chap:design}) de dicho motor, por lo que este objetivo también ha sido cumplido.

El tercer objetivo expone la necesidad de ampliar la interfaz gráfica para poder visualizar correctamente el nuevo tipo de registro. Así como la creación de nuevas pestañas y adaptar el sistema presente para cumplimentar este objetivo. Dada que la interfaz desarrollada (Sección~\ref{subsec:gui}) permite la visualización de registros tanto de control como vectoriales, facilitando la comprensión del modelo de programador, este objetivo ha sido cumplido.

El cuarto y último objetivo establece la necesidad de la facilitación de ejemplos que permitan entender el funcionamiento de la extensión. El conjunto de programas, aclaraciones y ejemplos del Capítulo~\ref{chap:v-extenssion}, junto con la variedad de test incluidos en el directorio \texttt{tests/riscv-vex/correct} suponen un conjunto amplio de ejemplos con los que cualquier usuario podría comprender la funcionalidad de cada instrucción. Es por esto que el objetivo se considera cumplido.

En conclusión, los objetivos planteados al inicio del proyecto han sido cumplidos. Lo que permite decir que el desarrollo de la propuesta ha resultado exitoso.

\subsection{Conclusiones personales}

En esta sección se elaboran unas breves conclusiones personales tras el desarrollo del proyecto.

El desarrollo del Trabajo de Fin de Grado presenta un reto que he debido afrontar individualmente, obligándome a enfrentarme al desarrollo completo de una propuesta personal. Esto ha hecho que ponga en práctica no sólo conocimientos técnicos adquiridos durante mi carrera académica. También me ha invitado a investigar y evaluar distintas tecnologías y procedimientos con el fin de alcanzar la excelencia en este trabajo.

Con lo anterior, destaco que un proyecto de estas características supone un reto. Pero también representa una oportunidad en la que un estudiante hace valer los conocimientos y capacidades que ha adquirido en los últimos años. Además, la posibilidad de trabajar en algo de libre elección y con gran autonomía crean un ambiente idóneo para que el estudiante termine de consolidar su formación académica y personal. Este tipo de proyectos no solo ponen a prueba sus habilidades técnicas, sino también su capacidad de organización, resolución de problemas y toma de decisiones. En definitiva, se convierten en una experiencia enriquecedora que marca un antes y un después en su trayectoria educativa.

En el caso concreto de este proyecto, la capacidad de entender el funcionamiento de un sistema y ampliar su funcionalidad ha resultado en un reto divertido, enriquecedor y que ha contribuido en gran medida a mi desarrollo como profesional.


\section{Trabajos futuros}\label{sec:future-work}

La herramienta desarrollada ofrece la capacidad de simular arquitecturas vectoriales a través del sistema CREATOR, tal y como se puso anteriormente. A continuación se exponen diferentes líneas de desarrollo que pueden continuar mejorando el sistema.

\begin{itemize}
    \item \textbf{Mejora del compilador}: como se ha visto durante este trabajo, las limitaciones del compilador ha impedido la implementación de las instrucciones de configuración y han obligado a añadir numerosas entradas para permitir emular parámetros enumerados. Una mejora sobre el compilador que tuviese en cuenta estas características facilitaría la definición de instrucciones y permitiría tener archivos de arquitecturas más simples y ligeros. En paralelo a la realización de este trabajo, se ha desarrollado un compilador mejorado para CREATOR de la mano de Álvaro Guerrero Espinosa~\cite{creatorcompiler}. No obstante, debido a ser un sistema en desarrollo de forma simultánea a este trabajo, ha sido imposible su inclusión.

    \item\textbf{Implementación de nuevas instrucciones}: actualmente se han implementado las instrucciones de memoria y aritmética entera. Para continuar dando soporte a la extensión V de RISC-V, un desarrollo futuro podría implementar otro tipo de instrucciones.

    \item\textbf{Elaboración de nuevas arquitecturas}: dado que sólo se ha deasrrollado un archivo de arquitecura para dar soporte a RISC-V, una mejora es el desarrollo de otras arqutiecturas vectoriales como AVX para x86 o NEON para ARM~(Capítulo \ref{chap:state-of-the-art}).

\end{itemize}

\section{Resumen}

En este último capítulo se han estudiado los objetivos propuestos durante la introducción, llegando a la conclusión de que se han cumplido. Además, se han incluido unas breves reflexiones personales acerca del proyecto y se han descrito distintos trabajos futuros que mejorarían la propuesta.
