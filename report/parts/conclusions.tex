\chapter{Conclusiones y trabajos futuros}\label{chap:conclusions}
En este capítulo se presentan las conclusiones tras la realización de este proyecto (Sección~\ref{sec:conclussions}). En esta sección se estudian los objetivos indicados en el capítulo~\ref{chap:introduction}. Además, se indican una serie de trabajos futuros (Sección~\ref{sec:future-work}) para la mejora de la propuesta.

\section{Conclusiones}\label{sec:conclussions}
En esta sección se hace un estudio de los objetivos propuestos en la sección~\ref{sec:objectives} con el fin de evaluar su cumplimiento.

El primer objetivo indica que se debe crear una nueva arquitectura para el simulador CREATOR con el nuevo conjunto de instrucciones implementado. Este objetivo se ha traducido directamente a un requisito de software (Capítulo~\ref{chap:analysis}) que ha sido verificado durante la validación y verificación del proyecto (Capítulo~\ref{chap:validation}). Por tanto, se puede afirmar que este objetivo se ha cumplido.

El segundo objetivo indica que se debe crear un motor de ejecución para dar soporte a operaciones con registros vectoriales. De nuevo, y al igual que en el caso anterior, esto se ha traducido a un requisito que ha sido validado en el correspondiente capítulo. Adicionalmente, se han expuesto los componentes y arquitectura (Capítulo~\ref{chap:design}) de dicho motor, por lo que este objetivo también se ha cumplido.

El tercer objetivo expone la necesidad de ampliar la interfaz gráfica para visualizar correctamente el nuevo tipo de registro, así como la creación de nuevas pestañas y adaptar el sistema de manera acorde. Dado que la interfaz desarrollada (Sección~\ref{subsec:gui}) permite la visualización de registros tanto de control como vectoriales, facilitando la comprensión del modelo de programador, este objetivo se ha cumplido.

El cuarto y último objetivo establece la necesidad de facilitar ejemplos que permitan entender el funcionamiento de la extensión. El conjunto de programas, aclaraciones y ejemplos del Capítulo~\ref{chap:v-extenssion}, junto con la variedad de test incluidos en el directorio \texttt{tests/riscv-vex/correct} suponen un conjunto amplio de ejemplos qeu facilitan la comprensión de cada instrucción. Es por esto que el objetivo se considera cumplido.

En conclusión, se han alcanzado todos los objetivos planteados  y, en consecuencia, cabe considerar un éxito el desarrollo de la propuesta.

\subsection{Conclusiones personales}

En esta sección se elaboran unas breves conclusiones personales tras el desarrollo del proyecto.

El desarrollo del Trabajo de Fin de Grado presenta un reto que he debido afrontar individualmente, obligándome a enfrentar el desarrollo completo de una propuesta personal. Esto ha requerido poner en práctica los conocimientos técnicos adquiridos durante mi carrera académica, además de llevarme a investigar y evaluar distintas tecnologías y procedimientos con el fin de alcanzar la excelencia en este trabajo.

Con lo anterior, destaco que un proyecto de estas características supone un reto, pero también una oportunidad para hacer valer los conocimientos y capacidades adquiridos. Además, la posibilidad de trabajar en un tema de libre elección y con gran autonomía crea un ambiente idóneo para consolidar la formación académica y personal. Este tipo de proyectos no solo ponen a prueba las habilidades técnicas, sino también la capacidad de organización, resolución de problemas y toma de decisiones. En definitiva, se convierten en una experiencia enriquecedora que marca un antes y un después en su trayectoria educativa.

En el caso concreto de este proyecto, la capacidad de entender el funcionamiento de un sistema complejo desarrollado por otras personas y ser capaz de ampliar con éxito su funcionalidad ha resultado en un reto estimulante, enriquecedor y que ha contribuido en gran medida a mi desarrollo como profesional.

\section{Trabajos futuros}\label{sec:future-work}

La herramienta desarrollada ofrece la capacidad de simular arquitecturas vectoriales a través del sistema CREATOR, tal y como se expuso anteriormente. A continuación se exponen diferentes líneas de desarrollo para mejorar el sistema.

\begin{itemize}
    \item \textbf{Mejora del compilador}: como se ha visto durante este trabajo, las limitaciones del compilador han impedido la implementación de las instrucciones de configuración y han obligado a añadir numerosas entradas para emular parámetros enumerados. Una mejora sobre el compilador que tuviese en cuenta estas características facilitaría la definición de instrucciones y permitiría archivos de arquitectura más simples y ligeros. En paralelo a la realización de este trabajo, se ha desarrollado un compilador mejorado para CREATOR de la mano de Álvaro Guerrero Espinosa~\cite{creatorcompiler}. No obstante, al tratarse de un desarrollo simultáneo, ha sido imposible su inclusión.

    \item\textbf{Implementación de nuevas instrucciones}: actualmente se han implementado las instrucciones de memoria y aritmética entera. Para continuar dando soporte a la extensión V de RISC-V, un desarrollo futuro podría implementar otro tipo de instrucciones.

    \item\textbf{Elaboración de nuevas arquitecturas}: una mejora sería el desarrollo de otras arquitecturas vectoriales adicionales como AVX para x86 o NEON para ARM~(Capítulo \ref{chap:state-of-the-art}).

\end{itemize}

\section{Resumen}

En este último capítulo se ha dejado constancia del cumplimiento de todos los objetivos del proyecto. Además, se ahn incluido unas breves reflexiones personales y se han enumerado diversos trabajos futuros que ampliarían el alcance de la propuesta.
