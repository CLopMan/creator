\chapter{Planificación y presupuesto}\label{chap:planning}
En este capítulo se presenta la planificación (Sección~\ref{sec:planning}) y presupuesto (Sección~\ref{sec:budget}) del proyecto. En la primera sección, se presenta la distribución de tareas y cronograma. En la segunda sección, se incluyen los costes del proyecto.

\section{Planificación}\label{sec:planning}
En esta sección se detalla la planificación seguida durante el desarrollo de la propuesta. En primer lugar, se realiza un breve análisis de las distintas metodologías (Sección~\ref{subsec:metodology}) de desarrollo con el fin de seleccionar justificadamente una de ellas. Después, se exponen las distintas fases del desarrollo del proyecto mediante un diagrama de Gantt (Sección~\ref{subsec:Gantt-diagram}).

\subsection{Análisis y selección de metodología}\label{subsec:metodology}
Se han valorado dos metodologías para la realización de este proyecto. A continuación se hace una breve introducción a cada una de ellas.

\begin{itemize}
    \item \textbf{Modelo en cascada~\cite{cascade}:} Este modelo se basa en fases secuenciales. Es decir, una fase no comienza hasta que termine la anterior. Se suele emplear para proyectos sencillos y de corta duración ya que carece de flexibilidad para realizar cambios sobre fases abnteriores.
    \item \textbf{Modelo en espiral~\cite{spiral}:} Este modelo se basa en iteraciones con fases definidas, de forma que el desarrollo se lleva a cabo de forma incremental, permitiendo realizar cambios entre iteraciones. Este modelo es adecuado para proyectos grandes y complejos, donde se puede sacar ventaja de la flexibilidad durante un largo periodo de desarrollo.
\end{itemize}

Finalmente, dadas las características del proyecto, se ha seleccionado el modelo de desarrollo en cascada. Los objetivos de la propuesta son claros y el alcance es asumible para poder realizarse sin necesidad de varias iteraciones. El desarrollo se ha dividido en las siguientes fases:
\begin{enumerate}
    \item \textbf{Planificación}: lectura y comprensión del estándar a implementar con el fin de extraer toda la funcionalidad necesaria para la implementación del conjunto de instrucciones seleccionado.
    \item \textbf{Análisis}: estudio del funcionamiento de CREATOR y evaluación de alternativas para satisfacer los objetivos fijados.
    \item \textbf{Desarrollo del motor de ejecución e implementación de instrucciones}: condificación y prueba de la funcionalidad extraída en las primeras dos fases.
    \item \textbf{Desarrollo y evaluación de la interfaz gráfica}: ampliación de la interfaz gráfica para la visualización de registros vectoriales.
    \item \textbf{Evaluación del sistema completo}: conjunto de pruebas de validación.
\end{enumerate}

\subsubsection{Estimación de tiempo y cronograma}\label{subsec:Gantt-diagram}
A continuación se muestra un diagrama de Gantt (Figura~\ref{Gantt-diagram}). Este diagrama representa cada una de las fases del proyecto junto con su extensión. Cada fase se ha dividido en varias tareas y subtareas con el fin de permitir una estimación más precisa del tiempo de desarrollo de la propuesta.

\begin{landscape}
\begin{figure}
      \begin{ganttchart}[
            hgrid,
            % vgrid,
            time slot format=isodate,
            time slot unit=day,
            x unit=.075cm,
            y unit chart=.481cm,
            y unit title=.7cm,
            title label font=\footnotesize,
            group label font=\bf\scriptsize,
            bar label font=\scriptsize,
            milestone label font=\it\scriptsize,
            expand chart=\textwidth
          ]{2024-08-01}{2025-06-18}
    \gantttitlecalendar{year, month=shortname} \\

    \ganttgroup{I. Planificación}{2024-08-08}{2024-09-09} \\
        \ganttbar{Comprensión de las principales arquitecturas vectoriales}{2024-08-08}{2024-08-20}\\
        \ganttlinkedbar{Lectura y comprensión del estándar RISC-V V}{2024-08-21}{2024-09-09}\\
    \ganttgroup{II. Análisis}{2024-09-10}{2024-11-10} \\
        \ganttbar{Análisis del estado actual de CREATOR}{2024-09-10}{2024-10-10}\\
        \ganttlinkedbar{Elicitación de requisitos}{2024-10-11}{2024-10-31}\\
        \ganttlinkedbar{Estudio de alternativas para la implementación}{2024-11-01}{2024-11-10}\\
    \ganttgroup{III. Desarrollo del motor de ejecución}{2024-11-11}{2025-04-11} \\
        \ganttbar{Implementación de funcionalidad básica}{2024-11-11}{2024-12-20}\\
        \ganttlinkedbar{Implementación y prueba de instrucciones}{2024-12-20}{2025-04-05}\\
        \ganttlinkedbar{Verificación del motor de ejecución}{2025-04-06}{2025-04-11}\\
    \ganttgroup{IV. Desarrollo de la Interfaz Gráfica}{2025-04-12}{2025-05-01} \\
        \ganttbar{Ampliación de la interaz gráfica}{2025-04-12}{2025-04-25}\\
        \ganttlinkedbar{Verificación de la interfaz gráfica}{2025-04-26}{2025-05-02}\\
    \ganttgroup{V. Evaluación del sistema completo}{2025-05-02}{2025-06-01}\\
        \ganttbar{Test de validación}{2025-05-02}{2025-06-01}\\
    \ganttgroup{Documentación del proyecto}{2024-12-20}{2025-06-01}\\


    \ganttlink{elem0}{elem3}
    \ganttlink{elem3}{elem7}
    \ganttlink{elem7}{elem11}
    \ganttlink{elem11}{elem14}
    \end{ganttchart}
    
    
    \caption{Planificación en forma de diagrama de Gantt}\label{Gantt-diagram}
\end{figure}
\end{landscape}

\section{Presupuesto}\label{sec:budget}
Esta sección detalla el presupuesto del proyecto, basado en las estimaciones de tiempo expuestas en la sección~\ref{sec:planning}. La tabla~\ref{table:project-summary} resume las principales características del proyecto.

\begin{table}[H]
    \begin{tabular}{@{}ll@{}}
    \toprule
    \textbf{Titulo}          & Desarrollo de instrucciones vectoriales RISC-V para el simulador CREATOR \\ \midrule
    \textbf{Autor}           & César López Mantecón  \\ \midrule
    \textbf{Fecha de Inicio} & 10-08-2024            \\ \midrule
    \textbf{Fecha Final}     & 01-06-2025            \\ \midrule
    \textbf{Duración}        & 10 meses              \\ \midrule
    \textbf{Presupuesto}     & 10.704,11\euro        \\ \bottomrule
    \end{tabular}
    \caption{Información del proyecto}\label{table:project-summary} 
\end{table}

En las siguientes secciones se detallará un desglose de los costes del proyecto. Estos costes se dividirán en costes directos (sección~\ref{subsec:direct-cost}), aquellos relacionados con el desarrollo del proyecto; y costes indirectos (sección~\ref{subsec:indirect-cost}), aquellos derivados de la actividad de desarrollo, pero no directamente relacionados con el producto o servicio específico. Todos estos costes se detallan sin impuestos. Estos últimos se incluyen el la sección~\ref{subsec:offer}, \textit{Oferta Propuesta}.

\subsection{Costes directos}\label{subsec:direct-cost}
Los costes directos son aquellos directamente relacionados con el desarrollo de la propuesta. Estos se dividen en costes de personal y costes de equipamiento.

\begin{itemize}
    \item\textbf{Costes de personal:} costes directamente dependientes del rol, responsabilidad y tiempo de cada miembro del equipo de desarrollo.
    \item\textbf{Costes de equipamiento:} costes dependientes de las herramientas utilizadas durante el desarrollo del proyecto.
\end{itemize}

\subsubsection{Costes de personal}
Para el desarrollo de la propuesta han sido necesario 4 roles: 
\begin{itemize}
    \item \textbf{Jefe de proyecto:} su principal responsabilidad es la coordinación del proyecto y llevar a cabo la planificación del mismo.
    \item \textbf{Analista}: su principal responsabilidad es la del análisis de requisitos de usuario y diseño de la arquitecura del proyecto.
    \item \textbf{Desarrollador:} su principal responsabilidad es la de la implementación de las funcionalidades indicadas en los requisitos de software.
    \item \textbf{Evaluador}: su principal responsabilidad es la de diseñar y ejecutar las pruebas del sistema, así como aportar feedback al equipo de desarrollo.
\end{itemize}

En la siguiente tabla se detalla el desglose de los costes de personal del proyecto.

% Estos costes se han estimado en base a la \textit{Encuesta cuatrienal de estructura salarial de 2022} realizada por el Instituto Nacional de Estadística (INE)~\cite{ine-salaries}.

\begin{table}[htbp]
    \centering
    \caption{Costes de personal del proyecto}
    \label{tab:costes_personal}
    \begin{tabular}{@{}lccc@{}}
    \toprule
    \textbf{Rol}     & \textbf{Horas} & \textbf{Coste por Hora} & \textbf{Total}   \\ \midrule
    Jefe de Proyecto & 30h            & 60\euro                 & 1.800\euro       \\
    Analista         & 80h            & 35\euro                 & 2.800\euro       \\
    Desarrollador    & 180h           & 30\euro                 & 5.400\euro       \\
    Evaluador        & 30h            & 20\euro                 & 600\euro         \\ \midrule
    \textbf{Total}   & 320h           &                         & \textbf{10.600\euro} \\ \bottomrule
    \end{tabular}
\end{table}

Finalmente, el coste total de personal es de 10.600\euro.

\subsubsection{Costes de equipamiento}
En este apartado se detalla el material necesario para el desarrollo del proyecto. La tabla~\ref{tab:material} muestra un desglose de los costes de equipamiento. Estos costes se han calculado en base a la siguiente fórmula:

\[C = \frac{c\times u\times t}{a}\]

Donde:
\begin{itemize}
    \item $C$: Coste de amortización del recurso
    \item $c$: Coste del recurso
    \item $u$: Uso del recurso (horas)
    \item $t$: Tiempo de amortización del recurso (horas)
    \item $a$: Amortización del recurso (horas)
\end{itemize}


\begin{table}[H]
    \begin{adjustbox}{max width=\textwidth}
    \centering
    \begin{tabular}{@{}lccccc@{}}
    \toprule
    \textbf{Concepto} & \multicolumn{1}{l}{\textbf{Coste (c)}} & \multicolumn{1}{l}{\textbf{Uso (u)}} & \multicolumn{1}{l}{\textbf{Tiempo usado (t)}} & \multicolumn{1}{l}{\textbf{Amortización (a)}} & \multicolumn{1}{l}{\textbf{Coste Amortizado (C)}} \\ \midrule
 Portátil       & 699,00\euro  & 50\%   & 10 meses     & 48 meses     & 65,53\euro     \\
 Monitor        & 149,95\euro  & 40\%   & 10 meses     & 60 meses     & 14,99\euro     \\
 Teclado        & 34,95\euro   & 80\%   & 10 meses     & 36 meses     & 5.82\euro   \\
 Silla Oficina  & 60,00\euro   & 60\%   & 10 meses     & 60 meses     & 4,00\euro   \\
 Software       & 0,00\euro    & 40\%   & 10 meses     & 120 meses    & 0,00\euro   \\ \midrule
 \textbf{Total} & 943,90\euro  &        &              &              & \textbf{90,35\euro}    \\ \bottomrule
    \end{tabular}
    \end{adjustbox}
    \caption{Costes de equipamiento del proyecto}\label{tab:material}
\end{table}

\subsection{Costes indirectos}\label{subsec:indirect-cost}
Los costes indirectos son aquellos derivados del desarrollo de la propuesta, pero no directamente relacionados con algún producto o servicio.

Para el cálculo de la energía consumida se tiene en cuenta que el portátil utiliza un promedio de 90W, y el teclado y monitor un promedio de 30W. Dado que el desarrollo ha durando 320h, la energía total consumida es de $120W\times 320h = 3840Wh$. El coste de promedio de la electricidad durante el periodo de desarrollo ha sido de 0.198\euro/KWh, lo que resulta en un coste por consumo eléctrico de $0.198*\frac{3840}{1000} = 0.76\euro$.

Adicionalmente, el coste en transporte al centro educativo se estima en un total de 13\euro imputables al proyecto. Esto supone un total de 13,76\euro.

\subsection{Resumen de costes}
Con todo lo anterior, se ha confeccionado la siguiente tabla con los costes del proyecto.

\begin{table}[H]
    \begin{tabular}{@{}lr@{}}
        \toprule
        Coste de personal & 10.600\euro\\
        Coste de equipamiento & 90,35\euro\\
        Costes indirectos & 13,76\euro\\ \midrule
        \textbf{Total} & \textbf{10.704,11\euro}\\
        \bottomrule
    \end{tabular}
    \caption{Costes del Proyecto}\label{tab:resumen-costes}
\end{table}

El coste final del proyecto asciende a \textbf{10.704,11\euro~(Diez mil setecientos cuatro euros con once céntimos)}.

\subsection{Oferta Propuesta}\label{subsec:offer}

A continuación se muestra la Tabla~\ref{tab:offer}, en la que se detalla la oferta propuesta. Se han estimado unos riesgos del 20\%, unos beneficios del 15\% y, por úlitmo, se han considerado unos impuestos del 21\% (atribuibles al Impuesto sobre el Valor Añadido en España). Tras aplicar los anteriores conceptos, el coste final del proyecto en caso de presentarse a un cliente es de \textbf{17.873,72\euro~(Diecisiete mil ochocientos setenta y tres euros con seteinta y dos céntimos)}.

\begin{table}[H]
    \begin{tabular}{@{}lccc@{}}
        \toprule
        \textbf{Concepto} & \textbf{Incremento} & \textbf{Coste Parcial} & \textbf{Coste} \textbf{Agregado}\\
        \midrule
        Coste del proyecto & -    & 10.704,11\euro & 10.704,11\euro \\
        Riesgo             & 20\% &  2.140,82\euro & 12.844,93\euro \\
        Beneficio          & 15\% &  1.926,74\euro & 14.771,67\euro \\
        Impuestos          & 21\% &  3.102.05\euro & 17.873,72\euro\\
        \midrule
        \textbf{Total}     & 56\% &        & \textbf{17.873,72\euro}\\
        \bottomrule
    \end{tabular}
    \caption{Oferta Propuesta}\label{tab:offer}
\end{table}

\section{Resumen}

En este capítulo se ha detallado la planificación y presupuesto de la propuesta. El tiempo estimado de desarrollo es de 320 horas, repartidas en 10 meses. En cuanto a los costes del proyecto, tras desglosarlos en costes directos (de personal y equipamiento) e indirectos (electricidad y transporte), se ha obtenido un coste final de 10.704,11\euro. Finalmente, se ha elaborado una oferta para el desarrollo de 17.873,72\euro.
