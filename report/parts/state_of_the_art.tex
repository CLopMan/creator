\chapter{Estado del arte}\label{chap:state-of-the-art}

Este capítulo presenta el conjunto de tecnologías actuales relacionadas con el
proyecto. Primero, se hace una breve introducción a las instrucciones vectoriales, sus
usos y las razones detrás de su auge; así como las principales arquitecturas
que las implementan. Después, se describen los principales simuladores de la
arquitectura RISC-V. Por último, se incluye una comparativa entre otros software de simulación y la
propuesta. Adicionalmente se incluye una breve descripción de las principales tecnologías WEB que influyen sobre este proyecto.

\section{Instrucciones vectoriales}\label{vector-instructions}

Las arquitecturas vectoriales operan con múltiples datos a través del uso de
grandes registros secuenciales. Una única instrucción es capaz de operar sobre
vectores de datos, lo que equivale a múltiples operaciones entre registros de
forma independiente. Esto presenta grandes ventajas, siendo capaces de superar
la latencia de la memoria al reducir las operaciones de carga y almacenamiento
de información, operando con grandes volúmenes de datos en el
procesador~\cite{hennessy2011computer}.

Tal y como ya se exponía en 1966 por Michael J. Flynn ~\cite{1447203}, muchos
problemas científicos de importancia requieren de un gran tiempo de
computación. Para abordarlo, propone una clasificación de los computadores que, según la naturaleza del problema, podrían presentar ventajas en la reducción del
tiempo de resolución:

\begin{itemize}
    \item \textit{Single instruction Stream-Single Data Stream (SISD)}
    \item \textit{Single instruction Stream-Multiple Data Stream (SIMD)}
    \item \textit{Multiple instruction Stream-Single Data Stream (MISD)}
    \item \textit{Multiple instruction Stream-Multiple Data Stream (MIMD)}
\end{itemize}

Las arquitecturas que dan soporte a instrucciones vectoriales son la
materialización de las unidades de procesamiento de tipo SIMD, pudiendo
realizar una operación sobre un conjunto de datos numeroso. Distintas
disciplinas con altas demandas computacionales y necesidad de operar con
grandes volúmenes de datos se ven beneficiadas por esta capacidad. Tres grandes
ejemplos son la Ciencia de Datos, el \textit{Machine Learning} y la Criptografía, tres áreas en
auge debido a la creciente cantidad de datos en la era actual, el interés en
la inteligencia artificial y los avances en el área de la Computación Cuántica y su impacto sobre los algoritmos de cifrado actuales. A continuación, se exponen las principales
alternativas de arquitecturas vectoriales existentes en la actualidad: AVX en
x86, ARM v8 y v9 y RISCV-V.

\subsection{Arquitectura x86}
La arquitectura \textit{x86} es la más adoptada en el mundo de los ordenadores
de sobremesa, cuya cuota de mercado se reparte casi en su totalidad entre los
fabricantes \href{https://www.intel.com}{Intel} y
\href{https://www.amd.com}{AMD}~\cite{tiwari2022amd}. Nace en 1978 con el
lanzamiento del procesador \textit{Intel 8086}~\cite{intel-8086} como una
arquitectura CISC (\textit{Complex Instruction Set Computer}) de propiedad de
Intel.

El desarrollo de las instrucciones vectoriales de esta arquitectura incluye el
lanzamiento y extensión de diferentes conjuntos de instrucciones como MMX, SSE,
SSE2, SSE3 y AVX~\cite{vector_research}. Donde AVX-512 es la versión más
reciente de la arquitectura.

AVX-512 es un conjunto de instrucciones capaz de acelerar el rendimiento del
procesamiento de vectores de datos~\cite{intel-avx512}. Estas instrucciones,
entre otras mejoras, doblan el ancho de registro de su predecesor (ahora de 512
bits). Esta tendencia a doblar el ancho de registro manteniendo el número de
unidades de procesamiento parece presentar una mayor eficiencia energética de
acuerdo con autores como~\cite{hennessy2011computer}.

Cabe destacar que AVX-512 no es un único \textit{Instruction Set Architecture} (ISA, en adelante), sino una familia de arquitecturas
que incluyen mejoras tanto el el hardware como en el set de instrucciones,
donde AVX-512F es el conjunto básico (también llamado AVX-512
\textit{Foundation}). Esta familia de arquitecturas tiene como objetivo la
aceleración de los procesadores para trabajar con grandes volúmenes de datos,
teniendo su foco principal en aplicaciones de simulación científica, Ciencia de
Datos o \textit{Machine Learning}, tal y como se aprecia en los casos de
estudio expuestos a continuación.

En~\cite{avx-ibm} se describe cómo el uso de AVX-512 ha permitido acelerar las
consultas del sistema \textit{watson.data}, una parte de \textit{IBM's Enterprise AI
solutions}, por un factor de 2,7 usando los procesadores Intel Xeon de quinta
generación. También se logra un factor de aceleración de 2,17 usando los Intel
Xeon de cuarta generación. Esto muestra el potencial que tienen esta clase de
arquitecturas para el tratamiento de grandes volúmenes de datos, especialmente
en el campo de la Inteligencia Artificial.

En~\cite{avx-cmcc} se expone cómo la compañía CMCC empleó esta clase de
arquitectura para construir un nuevo supercomputador para la simulación de
modelos de cambio climático. Para ello usó los procesadores Intel Xeon CPU Max
9480, logrando un factor de mejora de 3,47. Este supercomputador, apodado
Cassandra, no sólo mejoró el rendimiento de su predecesor (Zeus), sino que
también logró reducir el consumo eléctrico.

Además de un mayor ancho de registro, estos conjuntos de instrucciones ofrecen
una gran variabilidad de capacidades de acuerdo con~\cite{vector_research} y~\cite{intel-avx512-2}:

\begin{itemize}
    \item En la arquitectura base (AVX-512F): 
    \begin{itemize}
        \item Soporte de la funcionalidad anterior para un mayor tamaño de vector (512 bits).
        \item 32 registros vectoriales de 512 bits de ancho, 8 registros de máscara dedicados.
        \item Operaciones de 512 bits de elementos en coma flotante o enteros.
        \item Retrocompatibilidad con otras versiones de AVX sin penalización.
    \end{itemize}
    \item A través de extensiones:
    \begin{itemize}
        \item AVX-512VL\@: Capacidad de operar con vectores de longitudes menores (128 y 256 bits).
        \item AVX-512\_VNNI\@: Instrucciones especificas para la aceleración de redes neuronales convolucionales.
        \item VAES y GFNI\@: Aceleración por hardware para el algoritmo AES y operaciones en campos de Galois, dando soporte directo a operaciones criptográficas.
    \end{itemize}
\end{itemize}

Sin embargo, este conjunto de arquitecturas presentan los problemas comunes de
cualquier ISA CISC\@. El alto número de instrucciones hace complicada la
labor de los compiladores. Además, la escasez de licencias para su desarrollo
la hace poco adaptable a otra clase de sistemas fuera de las líneas de negocio
Intel y AMD\@.

En resumen, las arquitecturas AVX-512 presentan fuertes ventajas en el campo de
la aceleración de aplicaciones de cálculo intensivo, especialmente en los
campos de la simulación, Ciencia de Datos e Inteligencia Artificial. No
obstante, su naturaleza privativa hace de su desarrollo algo limitado a unas
pocas compañías, lo que limita su adaptabilidad y uso en componentes más variados
o con otra clase de necesidades.

\subsection{ARM}

ARM (Advanced RISC Machine) es una familia de ISA's basada en la filosofía de
los conjuntos de instrucciones reducidas (RISC) y desarrollado por la compañía
ARM Holdings. Este diseño permite a los procesadores contar con menor cantidad
de transistores que arquitecturas CISC como x86, lo que se traduce en un menor
coste, menores temperaturas y menor consumo de potencia~\cite{kaushikcase}.
Esto le ha llevado a ser líder en sectores como el IoT o los dispositivos
móviles.

En cuanto a su nacimiento, esta arquitectura surge en 1985 con el lanzamiento
del ARM1, un procesador de 32 bits enfocado en la simplicidad y eficiencia
energética~\cite{arm-history}. Sin embargo, no es hasta 1990 que se funda la
compañía ARM Holdings. Desde entonces, ARM Holdings ha desarrollado la
arquitectura, ofreciendo actualizaciones sobre el set de instrucciones y
unidades lógicas como \textit{IP-cores}~\cite{arm-history2}.

Durante sus múltiples actualizaciones se ha dando soporte a mayor cantidad
de instrucciones vectoriales, destacando las últimas versiones ARM v8 y ARM v9.
El groso de las instrucciones vectoriales de ARM son las instrucciones NEON\@.
NEON provee de instrucciones para el procesado de datos y la carga y
almacenamiento de los mismos. Estas instrucciones ofrecen una gran variedad de
operaciones aritmético-lógicas para enteros y control de flujo de datos entre
registro y memoria~\cite{arm-neon}\cite{arm-instructionsNEON}\@. Adicionalmente, la
arquitectura cuenta con el conjunto de instrucciones VFP, que añaden
operatividad para números en coma flotante~\cite{arm-vfp}.

Como ejemplo de lo que se puede lograr empleando esta arquitectura tenemos la
propuesta de Silvan Streit y Fabrizio De Santis~\cite{armv8-keyExchange}. En
este artículo se propone una implementación basada en ARM v8 con NEON para el
intercambio de claves a través de los protocolos NEWOHOPE y NEWHOPE-SIMPLE,
protocolos resistentes a la computación cuántica. Esta implementación logra una
mejora de rendimiento significativa, reduciendo el tiempo de ejecución con
respecto a la implementación en C de referencia en un factor de 8,3.
Adicionalmente, el tiempo invertido en el intercambio de clave se reduce
significativamente.

Con todo lo anterior, la arquitectura ARM ofrece instrucciones más genéricas y
en menor cantidad, facilitando la compilación el desarrollo para estas
plataformas. Adicionalmente, sus características de consumo y adaptabilidad la
hacen la opción predilecta para plataformas móviles o dispositivos IOT,
dos tecnologías ampliamente usadas. El uso de instrucciones vectoriales y su
aplicación en seguridad, criptografía y multimedia aporta un gran valor a esta arquitectura.
No obstante, su desarrollo vuelve a quedar sujeto a una única compañía que
limita el diseño de nuevos bloques lógicos basados en ARM.

\subsection{RISC-V V extension}

RISC-V es un conjunto reducido de instrucciones abierto y libre de
regalías~\cite{riscV-org}. Esta arquitectura nace en 2010 de la mano de los
profesores Krste Asanović y David Patterson y sus alumnos Yunsup Lee y Andrew
Waterman, cuando publican el primer volumen describiendo el conjunto de
instrucciones~\cite{riscv-2011}. El desarrollo de esta arquitectura tiene como
pilar las bases del código abierto, manteniendo un fuerte compromiso hasta la
actualidad con la libertad de diseño, elección y flexibilidad. Tanto el
conjunto de instrucciones básico como cualquier extensión o diseño lógico que haya sido desarrollado bajo el amparo de RISC-V International (organización sin ánimo de lucro
que alberga el estándar) están publicados bajo licencias abiertas aceptadas
globalmente~\cite{riscV-org}~\cite{asanovic2014instruction}.

RISC-V da soporte a instrucciones vectoriales a través de la extensión V de la
ISA, ratificada noviembre de 2021 e incluida por primera vez en la versión
20240411~\cite{riscv-isa2024} de la especificación de la
arquitectura~\cite{riscv-ratificationEspects}. Estas
instrucciones se caracterizan por ser ampliamente genéricas, dotando al
programador de mucha flexibilidad a la hora de desarrollar código en ensamblador.

Analizando brevemente la especificación~\cite{riscv-isa2024}, la arquitectura
para dar soporte a las instrucciones vectoriales añade 39 registros al conjunto
básico de instrucciones, 7 registros de control y 32 registros vectoriales. Los
registros vectoriales se identifican como $v{i}$ donde $i$ es una numeración
del 0 al 31. Los registros de control tienen nombres propios que se especifican
a continuación:

\begin{itemize}
    \item vstart: posición de inicio de un vector.
    \item vxsat: \textit{saturate flag} para aritmética de punto fijo.
    \item vxrm: modo de redondeo para la aritmética de punto fijo.
    \item vcsr: control y estado para registros vectoriales.
    \item vl: número de elementos de un vector.
    \item vtype: valores \texttt{vill}, \texttt{SEW}, \texttt{vmul},
    \texttt{vta} y  \texttt{vma}. Estos valores son parámetros de la
    arquitectura que representan la presencia de una configuración no
    permitida, el ancho de un elemento en bits, el factor de agrupamiento de
    los registros vectoriales y el comportamiento de los elementos de cola y
    enmascarados, respectivamente.
    \item vlenb: longitud del vector en \textit{bytes}.
\end{itemize}

En cuanto a las instrucciones, la extensión ofrece 4 grupos de instrucciones para operar con vectores de datos:
\begin{itemize}
    \item Instrucciones LOAD\@: dan soporte al vuelco en registros vectoriales de valores almacenados en memoria.
    \item Instrucciones STORE\@: dan soporte al vuelco en memoria de valores en los registros vectoriales.
    \item Instrucciones Aritméticas: dan soporte a operaciones
    arimético-lógicas entre vectores, entre vectores y escalares inmediatos y
    entre vectores y escalares en registro. Estas operaciones tienen sus
    variantes para enteros, punto fijo y coma flotante.
    \item Instrucciones de configuración: permiten cambiar los parámetros de
    vtype y algunos registros, como vl, en tiempo de ejecución.
\end{itemize}

Como ejemplo de lo que esta arquitectura es capaz, en el campo de las
comunicaciones existe una gran capacidad de paralelización a nivel de datos.
Tal y como se muestra en~\cite{riscv-studyCase}, el uso de vectorización en el
campo de las telecomunicaciones a través de RISC-V tiene un gran potencial para
la optimización de estos procesos. En la propuesta logran mejorar la
implementación basada en aritmética escalar en un factor de 60.

Con todo lo anterior RISC-V, pese a su relativa juventud, ofrece una
arquitectura sencilla y accesible que permite una flexibilidad para el
desarrollo de soluciones específicas. La ausencia de licencias y regalías hace
del diseño de arquitecturas algo accesible a cualquier particular o compañía,
destacando en campos como la investigación y la enseñanza.

\section{Simuladores RISC-V}

A continuación se hace un breve repaso por algunos simuladores de la
arquitectura RISC-V tanto en el ámbito de desarrollo como educativo.

\subsection{SPIKE}

Spike~\cite{spike} es un simulador de arquitectura RISC-V desarrollado en \texttt{C++}
por RISC-V International. Se trata del simulador de
referencia para la arquitectura RISC-V. Este da soporte a numerosas extensiones,
entre ellas la extensión de instrucciones vectoriales.  Adicionalmente, tiene
integración con GDB~\cite{gdb} para depurar código compilado con máquinas RISC-V
como objetivo.

En cuanto a su uso, se trata de una herramienta que se ha de ejecutar en local
a través de una interfaz de comandos. Al tener un enfoque en el desarrollo y
ampliación de la arquitectura se necesita un conocimiento especializado para su
uso.

\begin{lstlisting}[caption=Ejecución del simulador SPIKE]
# ejecución con spike
root@3407d21fb28b:/opt/riscv/test# spike pk hello
Hello   world!

# ejecución con spike en modo debug
root@3407d21fb28b:/opt/riscv/test# spike -d pk hello
[...]
core   0: 0x0000000080003464 (0xbe07b783) ld      a5, -1056(a5)
core   0: 0x0000000080003468 (0xfe078ce3) beqz    a5, pc - 8
core   0: 0x0000000080003460 (0x0000b797) auipc   a5, 0xb
core   0: 0x0000000080003464 (0xbe07b783) ld      a5, -1056(a5)
\end{lstlisting}

\subsection{SAIL}

SAIL es un lenguaje de descripción de ISA's caracterizado por 7 objetivos~\cite{sail}: 
\begin{itemize}
    \item Dar soporte a la semántica necesaria para la descripción de instrucciones.
    \item Ser accesible para el personal especializado, con la intención de
    parecerse a otros pseudocódigos en el mercado.
    \item Combinar una descripción de ISA secuencial con un modelo de memoria relajado propio.
    \item Desarrollar un sistema capaz de realizar verificaciones estáticas
    sobre el \textit{bitvector}\footnote{Un bit vector es \textit{array} de
    bits que representa una instrucción en memoria}, detectar de errores y
    generar de código, e inferir el tipo de dato.
    \item Dar soporte a emulación completa de la arquitectura basada en la definición de la instrucción.
    \item Soportar la generación automática de definiciones de probadores de
    teoremas con el fin de facilitar el razonamiento mecánico sobre la
    especificación de la ISA\@.
    \item Tratar de ser lo más minimalista posible.
\end{itemize}

Además, RISC-V ha adoptado este lenguaje para la descripción de su
ISA~\cite{sail-riscv}.

Este lenguaje cuenta con dos motores de ejecución oficiales, uno escrito en
\texttt{C} y otro escrito en \texttt{OCaml}. Ambas son herramientas basadas en
interfaz de comandos. Dada la naturaleza del lenguaje, su objetivo principal es
ayudar a desarrolladores y expertos en el campo de la Ciencia de la
Computación para facilitar la tarea de definir conjuntos de
instrucciones, por lo que se trata de una herramienta poco accesible para gente
con menos experiencia o alumnos en los primeros años de su carrera académica.

\begin{lstlisting}[caption=Ejecución del simulador sail-riscv, label={lstl:sail}]
build/c_emulator/riscv_sim_rv64d test/riscv-tests/rv64mi-p-access.elf
Running file test/riscv-tests/rv64mi-p-access.elf.
ELF Entry @ 0x80000000
tohost located at 0x80001000
begin_signature: 0x80002000
end_signature: 0x80002000
CSR mstatus <- 0x0000000A00000000 (input: 0x0000000000000000)
mem[X,0x0000000000001000] -> 0x0297
mem[X,0x0000000000001002] -> 0x0000
[0] [M]: 0x0000000000001000 (0x00000297) auipc t0, 0x0
x5 <- 0x0000000000001000
mem[X,0x0000000000001004] -> 0x8593
mem[X,0x0000000000001006] -> 0x0202
[1] [M]: 0x0000000000001004 (0x02028593) addi a1, t0, 0x20
x11 <- 0x0000000000001020
mem[X,0x0000000000001008] -> 0x2573
mem[X,0x000000000000100A] -> 0xF140
[2] [M]: 0x0000000000001008 (0xF1402573) csrrs a0, mhartid, zero
CSR mhartid -> 0x0000000000000000
[...]
mem[X,0x0000000080000046] -> 0xFC3F
[115] [M]: 0x0000000080000044 (0xFC3F2023) sw gp, -0x40(t5)
htif[0x0000000080001000] <- 0x00000001
htif-syscall-proxy cmd: 0x000000000001
SUCCESS
\end{lstlisting}

\subsection{CREATOR}\label{subsec:creator}

CREATOR (\textit{didaCtic and geneRic assEmbly progrAmming simulaTOR}) es una
plataforma web educativa e interactiva para la programación en
ensamblador desarrollada por el grupo de investigación \textit{ARCOS} de la
Universidad Carlos III de Madrid~\cite{camarmas2024creator}. Inicialmente
CREATOR nace como un
simulador para la enseñanza de lenguaje ensamblador, dando soporte únicamente
para la ISA MIPS32. No obstante, a día de hoy integra también RISC-V y permite
la definición y modificación de arquitecturas por los usuarios. Adicionalmente,
CREATOR cuenta con integración con dispositivos de hardware programables, haciendo de
la herramienta un entorno de desarrollo completo, educativo e intuitivo para la enseñanza
de código ensamblador.

Actualmente, CREATOR cuenta con diferentes funcionalidades que lo convierten en
una gran herramienta para la enseñanza: visualización de memoria y registros,
integración con hardware, ejecución paso a paso, soporte para puntos de ruptura
y carga de ficheros, entre otras. Además, CREATOR permite dos modos de
operación: a través de la interfaz web (Figura~\ref{fig:creator.jpg}) o a
través de la línea de comandos (Código~\ref{lst:creator-cli}).

\rasterfigure[0.85]{creator.jpg}{Interfaz gráfica de CREATOR}
\vfill
\vspace{0.5cm}
\begin{lstlisting}[caption=Interfaz por línea de comandos de CREATOR, label={lst:creator-cli}, language=bash]
creator> ./creator.sh -a architecture/RISC_V_RV32IMFD.json -s test/riscv/correct/examples/test_riscv_example_001.s

CREATOR
-------
version: 4.1
website: https://creatorsim.github.io/

[test/riscv/correct/examples/test_riscv_example_001.s]
12
34
[Architecture] Architecture 'architecture/RISC_V_RV32IMFD.json' loaded successfully.
[Library] Without library
[Compile] Code 'test/riscv/correct/examples/test_riscv_example_001.s' compiled successfully.
[Execute] Executed successfully.
[FinalState] cr[PC]:0xfffffffe; ir[x10,a0]:0x22; ir[x17,a7]:0x1; keyboard[0x0]:''; display[0x0]:'1234';
\end{lstlisting}

Además, su tecnología web lo hace portable y accesible desde cualquier
dispositivo sin necesidad de descargar o compilar algún elemento. Estas
características hacen de CREATOR una herramienta que ha resultado
verdaderamente útil en entornos educativos. Prueba de esto es su reconocimiento
por la \textit{RISC-V Foundation}~\cite{creator-riscvFoundation}, apareciendo
el primero en la lista de simuladores y emuladores de la arquitectura.


\section{Comparativa}

Con todo lo anterior es evidente que CREATOR y su integración con RISC-V
ofrecen un entorno didáctico y ampliable con numerosas ventajas para el
aprendizaje. Su sencillez, amigable interfaz y sus funcionalidades aportan un
gran valor a la experiencia de los alumnos de asignaturas como
\textit{Estructura de Computadores} y \textit{Arquitectura de Computadores}.
Sin embargo, está limitado al set básico de instrucciones de RISC-V, lo
que le aleja de alternativas más punteras y especializadas como SAIL o SPIKE\@.
Añadir soporte para instrucciones vectoriales, dado el contexto tecnológico
actual en el que múltiples disciplinas como la Ciencia de Datos, la Inteligencia Artificial o la Criptografía
están en auge, permitiría a los ingenieros en formación comprender conceptos de gran importancia en estas áreas
como son el procesamiento en paralelo y la aceleración de tareas mediante
extensiones de propósito específico. Esto no solo enriquecería su aprendizaje,
sino que también los prepararía mejor para enfrentar desafíos actuales en áreas
como el procesamiento masivo de datos, aprendizaje profundo y simulaciones
científicas. Además, implementar este soporte en CREATOR fomentaría la adopción
de RISC-V en ámbitos académicos y de investigación, consolidándose como una
plataforma versátil y relevante en el desarrollo de aplicaciones modernas.

A continuación se muestra una tabla comparativa de los diferentes simuladores presentados anteriormente:

\begin{table}[H]
\begin{tabular}{@{}ccccc@{}}
    \toprule
                  & Spike        & SAIL                        & CREATOR & Propuesta    \\
    \midrule
    Lenguaje      & \texttt{C++} & \texttt{C} / \texttt{OCaml} & \texttt{\js}        & \texttt{\js} \\
    V Extension  & $\checkmark$ &       $\checkmark$          &                     & $\checkmark$        \\
    GUI           &              &                             &    $\checkmark$     & $\checkmark$        \\
    CLI           & $\checkmark$ &       $\checkmark$          &    $\checkmark$     & $\checkmark$        \\
    IDE integrado &              &                             &    $\checkmark$     & $\checkmark$        \\
    En navegador  &              &                             &    $\checkmark$     & $\checkmark$        \\
    Interactivo   &              &                             &    $\checkmark$     & $\checkmark$        \\
    \bottomrule
\end{tabular}
\caption{Comparativa entre simuladores}\label{table:comparativa}
\end{table}

Como se puede ver, la propuesta pretende ampliar CREATOR, haciéndolo un
simulador más completo y poniéndolo a la altura de otros simuladores de la
arquitectura RISC-V.

\section{Tecnologías WEB}

Las tecnologías web permiten una mayor accesibilidad de forma independiente a
las características de cada dispositivo. Estas tecnologías permiten ofrecer
recursos digitales para que puedan ser utilizados por una gran variedad de
dispositivos.

En el ámbito de desarrollo de la propuesta, al tratarse de una extensión sobre
CREATOR, se trabaja con distintas tecnologías que se describen a continuación.

\subsection{HTML5}
HTML5 (\textit{Hyper Text Markup Language} 5)~\cite{html5-standard} es la última versión del lenguaje HTML, ampliamente utilizado para estructurar el contenido de las páginas web. Este lenguaje utiliza etiquetas y atributos para indicar al navegador el contenido que debe mostrar. Es decir, una página web desarrollada en HTML sólo contendrá texto y referencias externas. 

Este lenguaje ha tenido varias versiones a lo largo del tiempo, recogidas en diferentes estándares. El desarrollo de estos documentos es llevado a cabo por la institución \textit{World Wide Web Consortium} (W3C)~\cite{www-consortium}\@. La versión 5 incluye más de 100 etiquetas y, frente a su versión anterior, da soporte a diferentes archivos multimedia (audio y vídeo)~\cite{htm4vshtml5}\@.

\subsection{CSS}

CSS (\textit{Cascade Style Sheets}) es un lenguaje para hojas de estilo usado para describir la forma de presentar el contenido definido en un archivo HTML o XML~\cite{css}\@. Este lenguaje complementa a HTML5, permitiendo separar el contenido de la página web de su presentación. De la misma forma que en el caso de HTML5, la especificación y desarrollo de este lenguaje se lleva a cabo por la institución W3C.

El lenguaje CSS permite definir y comunicar al navegador diferentes características relativas a la presentación del contenido como familias de fuentes, colores, estilos o tamaños. Además permite la distinción de presentaciones en base a características como el tamaño de la pantalla, lo que permite crear páginas atractivas y adaptadas a cualquier dispositivo (\textit{responsive}). 

De esta forma, el lenguaje permite crear páginas web con una mejor estructura. La separación entre presentación y contenido presenta claras ventajas como la reutilización de estilos para distintos componentes de la página web o la simplificación de archivos HTML\@.

\subsection{\js}
{\js} es un lenguaje de programación interpretado y ligero. Se trata del lenguaje más utilizado en el ámbito del desarrollo web ya que se soporta de forma nativa por la mayoría de navegadores~\cite{javascript}\@. Este trabaja conjuntamente con CSS y HTML5, aportando la funcionalidad a la página web.

Este lenguaje adopta una sintaxis parecida a la de C o C++, para reducir el esfuerzo en su aprendizaje, y está fuertemente ligado a la programación orientada a objetos. Además, se trata de un lenguaje de \textit{scripting}, lo que significa que el código puede insertarse dentro de código HTML o en un archivo externo. El navegador web ejecutará el código {\js} en el lado cliente, sin hacer uso de un servidor.

{\js} está estandarizado en el EMAScript Language Specification~\cite{EMACScript}, el estándar utilizado por los navegadores web modernos. Es por esto que {\js} es ampliamente usado para el desarrollo de aplicaciones web.

De forma complementaria, ha surgido una nueva aproximación a la funcionalidad web de la mano con otros lenguajes de programación como C++ o Rust, \wa. {\wa} es un tipo de código que permite la ejecución de código escrito en C, C++, Rust o C\# en navegadores web. Su diseño se ha orientado como un complemento a \js, no como un sustituto. Sin embargo, \wa permite solventar algunos problemas que presenta {\js}, destacando una mejora en el rendimiento. Más adelante se realizará una comparativa más profunda acerca de estas dos alternativas para dotar de funcionalidad a las páginas web (Sección \ref{subsec:programming-languages}).

\subsection{Vue.js}
Vue.js es un \textit{framework} de {\js} de código abierto para el desarrollo de interfaces de usuario en aplicaciones web. Este se basa en HTML, CSS y {\js}, y provee un modelo de programación basado en componentes~\cite{vuejs}. CREATOR adopta este \textit{framework} para la definición y desarrollo de su interfaz gráfica.

Vue.js se caracteriza por la capacidad de crear componentes y combinarlos para la creación de páginas web complejas. Estos componentes, más simples, son reutilizables e independientes. De esta forma, se permite el desarrollo de aplicaciones web de cualquier complejidad, separando las funcionalidades por componentes.

\section{Resumen}

Con todo lo anterior, se ha hecho un resumen de las principales tecnologías y áreas de desarrollo en el ámbito de la propuesta, dando una visión global del estado del arte relativo a las arquitecturas vectoriales y los principales simuladores RISC-V. Adicionalmente, se han detallado las principales tecnologías web que relacionadas con la propuesta: HTML5, CSS, {\js} y Vue.js\@.
