\documentclass[es]{uc3mthesisIEEE}

\usepackage[utf8]{inputenc}
\usepackage{import}
\usepackage{enumitem}  % control item separation -> \begin{itemize}[nosep]
\usepackage{textcomp} 
\usepackage{lipsum}  % dummy text
\usepackage{placeins}  % \FloatBarrier -> prevents figures and tables from passing that point
\usepackage{tikz}
\usetikzlibrary{arrows.meta, positioning}
\usepackage{circuitikz}
\usepackage[es,enableTraceability,enableCaptions]{srs}
\usepackage{tcolorbox}
\usepackage{dirtree}
\usepackage{floatrow}
\usepackage{pdflscape}
\usepackage{pgfgantt}

\usetikzlibrary{arrows.meta, positioning, shapes.geometric}
\usepackage{environ}
\makeatletter
\newsavebox{\measure@tikzpicture}
\NewEnviron{scaletikzpicturetowidth}[1]{%
  \def\tikz@width{#1}%
  \def\tikzscale{1}\begin{lrbox}{\measure@tikzpicture}%
  \BODY
  \end{lrbox}%
  \pgfmathparse{#1/\wd\measure@tikzpicture}%
  \edef\tikzscale{\pgfmathresult}%
  \BODY
}
\makeatother
\usepackage{mymacros}  % report-specific macros


% silence ht warnings
\usepackage{silence}
\WarningFilter{latex}{`h' float specifier changed to `ht'}


\graphicspath{{img/}}  % Images folder


% REFERENCES
\addbibresource{references.bib}  % bibliography file
%\import{}{glossary.tex}  % glossary file


\newcommand{\js}{JavaScript~}
\newcommand{\wa}{WebAssembly~}

%	DOCUMENT

% setup
\degree{Grado en Ingeniería Informática}
\title{Desarrollo de instrucciones vectoriales\\ \vspace{0.3cm} RISC-V para el simulador CREATOR}
\shorttitle{Extensión vectorial para CREATOR}
\author{César López Mantecón}
\advisors{Felix García Caballeira}
\place{Leganés, Madrid, España}
\date{Junio 2025}

\definecolor{mygreen}{rgb}{0.45, 0.85, 0.0}


\lstset{
    literate={á}{{\'a}}1
        {ã}{{\~a}}1
        {é}{{\'e}}1
        {ó}{{\'o}}1
        {í}{{\'i}}1
        {ñ}{{\~n}}1
        {¡}{{!`}}1
        {¿}{{?`}}1
        {ú}{{\'u}}1
        {Í}{{\'I}}1
        {Ó}{{\'O}}1
}

\begin{document}

  % COVER
  \makecover


  % EPIGRAPH
  \makeepigraph
    {El trabajo de uno reluce por sí mismo}  % quote
    {Beatriz Mantecón García}  % author
    {}  % source


  % ACKNOWLEDGEMENTS
  \begin{acknowledgements}
      Este trabajo no habría sido posible sin el apoyo de familiares, amigos y colegas que me han acompañado durante años. Especialmente, quiero hacer mención a mis padres: Beatriz Mantecón y José Manuel López. Personas capaces de aguantar contra viento y marea con tal de ver a cualquiera de los tres hermanos triunfar. Durante toda mi educación me han arropado y apoyado, velando por mi bienestar y mi educación. Quiero mencionar adicionalmente a mi abuela, Mercedes García, una de las mujeres más importantes de mi vida y que no tengo palabras para destacar lo mucho que la quiero. Por último, no puedo cerrar este párrafo sin mentar a Paula y Jaime, mis hermanos; a mis primos Miguel y Laura; y a mis tíos Patricia y Juan Andrés. Su ayuda, apoyo y cariño es algo que siempre tendré en mente.

      Otro fuerte pilar durante mi educación han sido mis amigos. Creo que estoy rodeado de grandes personas que se merecen reconocimiento. El grupo ``Pan con Pan'' hemos compartido disgustos y alegrías durante cuatro años, y he recibido mucho apoyo por su parte. Especialmente hablando de Adrián Fernández, creo que su sencillez, honestidad y valores le hacen una gran persona y harán de él un gran profesional.

      Por último, me gustaría hacer una mención especial para Jontxu y Carlitos. Estas dos personas me llevan acompañando casi desde que tengo memoria. Estos nombres representan, para mí, el verdadero significado de la palabra amigo.

  \end{acknowledgements}


  % ABSTRACT
  \begin{abstract}
      El presente Trabajo Fin de Grado tiene como objetivo principal la ampliación del simulador educativo CREATOR mediante la incorporación de soporte para instrucciones vectoriales de la arquitectura RISC-V. Este desarrollo se enmarca en un contexto tecnológico en el que las arquitecturas vectoriales han adquirido gran relevancia debido al auge de disciplinas como la inteligencia artificial y la ciencia de datos, que requieren una elevada capacidad de procesamiento paralelo y operaciones eficientes sobre grandes volúmenes de datos.

La motivación del proyecto surge de la necesidad de contar con herramientas didácticas accesibles, intuitivas y potentes que permitan a los estudiantes explorar conceptos fundamentales de arquitectura de computadores y programación a bajo nivel. CREATOR, desarrollado por el grupo de investigación ARCOS de la Universidad Carlos III de Madrid, ya se utiliza ampliamente en la enseñanza de ensamblador sobre arquitecturas como MIPS y RISC-V. Sin embargo, hasta la fecha, carecía de soporte para la extensión vectorial (extensión V) de RISC-V, lo que limitaba su potencial pedagógico en áreas más avanzadas.

El trabajo se divide en varias fases, comenzando por un análisis detallado del estado del arte, donde se examinan otras arquitecturas vectoriales como AVX (x86), NEON (ARM) y la propia extensión V de RISC-V. También se comparan diferentes simuladores como SPIKE y SAIL, destacando CREATOR por su accesibilidad web y enfoque educativo.

Posteriormente, se define la arquitectura vectorial a implementar, incluyendo parámetros como VLEN, SEW y LMUL, así como los registros de control específicos como VTYPE y VL. Se diseñó un motor de ejecución específico para las operaciones vectoriales, capaz de interpretar y ejecutar instrucciones como carga, almacenamiento, operaciones aritmético-lógicas y de configuración. Además, se ampliaron las funcionalidades de la interfaz gráfica de CREATOR, permitiendo visualizar el estado de los registros vectoriales y estadísticas de ejecución de forma clara y didáctica.

La implementación se llevó a cabo utilizando tecnologías web como JavaScript y Vue.js, asegurando la portabilidad y compatibilidad con la plataforma existente. A lo largo del proyecto, se realizaron exhaustivas pruebas de verificación y validación para garantizar la corrección funcional del sistema y su utilidad educativa. También se incluyó una planificación detallada y un estudio de costes, mostrando la viabilidad del proyecto desde una perspectiva técnico-económica.

Finalmente, se concluye que la incorporación de instrucciones vectoriales a CREATOR no solo mejora la funcionalidad del simulador, sino que representa una contribución valiosa a la enseñanza de arquitectura de computadores. Además, se proponen futuras líneas de trabajo, como la incorporación de nuevos tipos de instrucciones o el soporte para operaciones en coma flotante, lo cual seguiría enriqueciendo la herramienta.
    \keywords{RISC-V \sep CREATOR \sep Vector \sep Extension-V}
  \end{abstract}


  % TOC
  \tableofcontents
  \listoffigures
  \listoftables


  % THESIS
  \begin{thesis}
    \includefrom{parts/}{introduction.tex}
    \includefrom{parts/}{state_of_the_art.tex}
    \includefrom{parts/}{VectorExtenssionRiscV.tex}
    \includefrom{parts/}{analysis.tex}
    \includefrom{parts/}{design.tex}
    \includefrom{parts/}{implementation.tex}
    \includefrom{parts/}{validation.tex}
    \includefrom{parts/}{planning.tex}
    \includefrom{parts/}{regulation.tex}
    \includefrom{parts/}{conclusions.tex}
    \newpage
    %\import{parts/}{example.tex}
  \end{thesis}


  % BIBLIOGRAPHY
  \cleardoublepage
  \label{bibliography}
  \printbibliography[heading=bibintoc]


  % GLOSSARY
  %\cleardoublepage
  %\label{glossary}
  %\printglossaries
  % \printnoidxglossaries[type=\acronymtype]  % slower, but no need to do $ makeglossaries report


  % APPENDICES
  % \begin{appendices}
  %   \chapter{My stuff}
  %   \lipsum
  % \end{appendices}


\end{document}
